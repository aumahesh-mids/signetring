\section{Introduction}
\label{sec:intro}

The rampant growth of malicious \enquote{deepfakes} in the media has created a need for an \enquote{anti-disinformation} solution \cite{ckdg20}. Stories such as the {\em Grenfell Tower fire} \cite{grenfell-wiki}, where posts of missing or deceased individuals were created with photos of social media celebrities or writers who were not actually in the vicinity of the incident, have taken the news by storm \cite{bl18}. Unfortunately, there is no indication that incidents such as these will cease.

Deepfakes are just part of the problem. When it comes to reporting current events, journalists have confidential sources, but there is a constant push by consumers/readers (henceforth referenced as readers) who want to verify the origin of the sources in order to protect themselves from fake news or bad actors.

A possible solution would allow journalists/publishers to verify the origin of the digital objects their sources provide and allow sources to verify that their digital objects are shared with the intended journalist/publisher. This solution should implement a system with no agenda other than verifying the source of an object and maintaining an object's {\em lineage}. It would be uninterested in the content it verifies and tracks, providing some privacy to the sources who post within it. Furthermore, the system cannot prioritize one object over any other. Instead, it should allow sources and publishers to determine what is worth sharing while still tracking every change to the original object, thus making the ideal system unbiased.
Furthermore, if a vendor, such as a phone manufacturer, were to design and implement such a system, they may be motivated by proprietary requirements and company interests than the universality of use and user privacy. For example, a Google app would be interested in collecting user data and biased towards some content, sources, or publishers. 

Currently, an unbiased and uninterested system does not exist. To address this concern, we propose \name.\cprotect\footnote{{\em signet} .d: \enquote{a seal used officially to give personal authority to a document in lieu of signature} \cite{mw:signet}}$^{,}$
\footnote{{\em signet ring} .d: \enquote{a finger ring engraved with a signet, seal, or monogram} \cite{mw:signet-ring}}. \name allows the sources and the publishers to verify each other. It also provides a mechanism that protects owner-anonymity when a reader verifies a published object. Furthermore, it takes the burden of verification off the readers. Instead, news outlets must provide information/digital objects that are verifiable. 

\name provides robust confidentiality to the source of a digital object while still allowing a publisher to authenticate the source and the object through a trusted entity for authenticity. The protection applies to the digital objects created in a device connected to \name and the edits (e.g.,  adjusting lighting) made to that digital object. It verifies that the source is the owner of the object. Additionally, it allows a reader to verify that the published digital object is authentic and that its lineage is protected. 
