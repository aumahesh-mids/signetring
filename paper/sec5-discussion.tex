\section{Discussion}
\label{sec:discussion}

%In this section, we address some open questions about \name and identify the potential scope for future enhancement.

\subsection{Staged Objects}
\name ensures that when the \owner creates an object through an \app, the \app submits the object to the \ta and requests a certificate. The action of the creation of an object is the trigger for the request for a certificate. However, the architecture does not control the environment at the time of the creation of the object. As an example, consider the process of clicking a photograph using a camera app. When the user clicks, the \app shoots the image and submits it to the \ta for a certificate. However, the environment/context where the image was shot is beyond the purview of \name. \name only guarantees that an object submitted to \ta and published by a \publisher is authentic, i.e., it is certified at the time of the creation of the object. 

\subsection{Trustworthiness of Applications}
Staged objects raised the question about how to deal with applications or owners that are not trustworthy. Consider the following scenario. A third-party authenticated photo editing software submits an edited version of an already certified digital object without providing lineage. To deal with this problem, we propose using semantic similarity hashing (SSHash) \cite{sshash-1,sshash-2}, similar to architecture proposed in \cite{qqjs19}. Specifically, the \ta runs a semantic similarity search across the objects it has already certified. In the case where an existing certified object has a significant similarity match (above some predefined threshold) with the submitted object, \ta flags the object as a violation and does not issue a certificate. 

\subsection{Verification of Real Identities}
When the \ta registers a user, it has to ensure that the user is whom they claim to be. For example, consider a user who registers with the \ta as CNN or Fox News. \ta should not accept the request for registration without proper verification. Verifying the real identity is beyond the scope of this work and is usually a manual or a semi-automated process through verification vendors. Moreover, this verification of real identity is similar to verifying subject names by a certificate authority when a subject requests a digital certificate \cite{ca-validation}. Also, this is similar to how various social media platforms verify users' identities and add a verified icon to their profiles (e.g., Twitter Blue \cite{twitter-blue}). Therefore, \ta has to verify a user's real identity before accepting their registration request. 

\subsection{Anonymity in Challenge Protocol}
\label{sec:anon-challenge}
As mentioned in Section \ref{sec:pub-object}, the challenge protocol can be implemented in many different ways. In our implementation, for quick prototyping and demonstration of our approach, we let the \owner communicate with the \publisher directly during the challenge exchange protocol. Instead, we can implement the challenge protocol such that the \owner submits the challenge request for the \ta to manage. Specifically, the \owner initiates the challenge, and the \ta executes the challenge by forwarding the challenge to the requested \publisher. The \owner and the \publisher do not communicate directly and do not know the other party's public key. Instead, \ta manages the whole protocol and sends only the result to the respective parties. Thus, it is possible to publish a digital object anonymously. 

\subsection{Anonymity Everywhere}
In the current design, the \owner of a digital object has to authenticate with the \ta to get a certificate and publish the object. One obvious question is whether it is possible to protect the identity of the \owner from the \ta itself. One approach is to create virtual entities every time a digital object is created. The virtual entities include a virtual app and a virtual user, with no traceability to the original app and the \owner.
Nevertheless, registration and verification of such virtual entities is an open question. Current architecture requires the \ta to verify and authenticate the \app and the \owner. We do not yet know if the anonymous creation and publication of digital objects are possible. And this question is beyond the scope of this work. 

\subsection{Inference Threats}
All communication in \name is encrypted with appropriate keys negotiated as part of the TLS exchange \cite{RFC5246}. However, inference threats on encrypted traffic are still possible (as studied in \cite{wbmm07, mhjt14}). As a result, it is possible to make some inferences based on the communication between the various components and users of \name. To minimize such attacks, we propose to extend \name to implement the anonymity-preserving challenge protocol for publishing a digital object, as discussed in Section \ref{sec:anon-challenge}. However, a third party can still make inferences by just observing communication patterns between the \owner and the \ta, or the \ta and the \publisher. For example, an attacker could correlate the sequence of communication to infer that a particular \owner is trying to publish an object with a particular \publisher. To address this issue, \ta will bulk challenge messages for a publisher and dispatch them at a fixed time of the day, minimizing the correlation. In addition, we note that a given user, \app, and \ta communicate for various reasons (authentication, creation, verification, publication) over a channel created using a negotiated symmetric key between the respective parties. Thus, the possibility of any kind of inference is very low. 

\subsection{\name in Production}
The POC implementation discussed in Section \ref{sec:results} is very limited. To bring \name to production, we have the following options:

\begin{itemize}
    \item {\em End-to-end ecosystem. } We design and develop all components of the architecture. We would invest in the development of various content-creation applications such as camera and word processor. However, there are existing applications such as the native mobile camera app with huge adoption. As a result, getting consumers to adopt native \name content-generation applications will be the main challenge. 
    \item {\em Connectors. } We design connectors for existing third-party content-generation applications. Connectors allow \name to provide an unbiased and uninterested privacy-centric framework for such applications. Furthermore, to increase awareness of consumers to use applications that have connectors for \name, we plan to work with popular App Stores (e.g., Apple App Store, Google Play Store) to add a seal of approval (e.g., {\em Verified by \name}) to such applications. 
\end{itemize}

